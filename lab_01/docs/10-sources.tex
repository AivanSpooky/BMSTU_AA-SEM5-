\renewcommand{\bibname}{}
\begin{thebibliography}{6}
\renewcommand{\bibname}{СПИСОК ИСПОЛЬЗОВАННЫХ ИСТОЧНИКОВ}
\begin{center}
    \textbf{\bibname}
\end{center}
\addcontentsline{toc}{chapter}{СПИСОК ИСПОЛЬЗОВАННЫХ ИСТОЧНИКОВ}
    \bibitem{bib0}
    Levenshtein Distance - an overview [Электронный ресурс]. URL: \url{https://www.sciencedirect.com/topics/computer-science/levenshtein-distance} (дата обращения: 08.09.2024).
	\bibitem{bib1}
	Levenshtein Distance for Dummies [Электронный ресурс]. URL: \url{https://medium.com/analytics-vidhya/levenshtein-distance-for-dummies-dd9eb83d3e09} (дата обращения: 08.09.2024).
	\bibitem{bib2}
	Welcome to Python [Электронный ресурс]. URL: \url{https://www.python.org} (дата обращения: 08.09.2024).
	\bibitem{bib3}
	time — Time access and conversions [Электронный ресурс]. URL: \url{https://docs.python.org/3/library/time.html#functions} (дата обращения: 08.09.2024).
	\bibitem{bib4}
	Windows technical documentation for developers and IT pros [Электронный ресурс]. URL: \url{https://learn.microsoft.com/en-us/windows/} (дата обращения: 08.09.2024).
	\bibitem{bib5}
	Intel® Core™ i5-10300H Processor [Электронный ресурс]. URL: \url{https://ark.intel.com/content/www/us/en/ark/products/201839/intel-core-i5-10300h-processor-8m-cache-up-to-4-50-ghz.html} (дата обращения: 08.09.2024).
\end{thebibliography}