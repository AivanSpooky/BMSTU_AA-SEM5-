\chapter{Технологическая часть}

В данном разделе будут приведены требования к программному обеспечению, средства реализации, листинги кода.

\section{Требования к программному обеспечению}

Входные данные: две строки на русском или английском языке в любом регистре;

Выходные данные: искомое расстояние для выбранного метода и матрицы расстояний для матричных реализаций.

\section{Средства реализации}
В данной работе для реализации был выбран язык программирования $Python$ [3]. Выбор обсуловлен наличием функции вычисления процессорного времени в библиотеке \textit{time} [4]. Время было замерено с помощью функции \textit{process\_time()}.

\section{Реализация алгоритмов}

В листингах \ref{lst:lev_recursion} - \ref{lst:dam_lev_dyn} представлены реализации алгоритмов нахождения расстояния Левенштейна и Дамерау–Левенштейна.

\clearpage

\begin{center}
\captionsetup{justification=raggedright,singlelinecheck=off}
\begin{lstlisting}[label=lst:lev_recursion,caption=Функция нахождения расстояния Левенштейна рекурсивно]
def levenstein_recursive(str1, str2, output = True):
    n = len(str1)
    m = len(str2)

    if ((n == 0) or (m == 0)):
        return abs(n - m)

    if (str1[0] == str2[0]):
        return levenstein_recursive(str1[1:], str2[1:])

    min_distance = 1 + min(levenstein_recursive(str1[1:], str2),
                    levenstein_recursive(str1, str2[1:]),
                    levenstein_recursive(str1[1:], str2[1:]))

    return min_distance
\end{lstlisting}
\end{center}


\begin{center}
\captionsetup{justification=raggedright,singlelinecheck=off}
\begin{lstlisting}[label=lst:lev_table,caption=Функция нахождения расстояния Левенштейна динамически]
def levenstein_dynamic(str1, str2, output = True):
    n = len(str1)
    m = len(str2)

    matrix = create_lev_matrix(n + 1, m + 1)

    for i in range(1, n + 1):
        for j in range(1, m + 1):
            add = matrix[i - 1][j] + 1
            delete = matrix[i][j - 1] + 1
            change = matrix[i - 1][j - 1]
            
            if (str1[i - 1] != str2[j - 1]):
                change += 1

            matrix[i][j] = min(add, delete, change)

    if (output):
        print_lev_matrix(matrix, str1, str2)

    return matrix[n][m]
\end{lstlisting}
\end{center}

\clearpage

\begin{center}
\captionsetup{justification=raggedright,singlelinecheck=off}
\begin{lstlisting}[label=lst:dam_lev_dyn,caption=Функция нахождения расстояния Дамерау–Левенштейна динамически]
def damerau_levenstein_dynamic(str1, str2, output=True):
    n = len(str1)
    m = len(str2)

    matrix = create_lev_matrix(n + 1, m + 1)

    for i in range(1, n + 1):
        for j in range(1, m + 1):
            add = matrix[i - 1][j] + 1    
            delete = matrix[i][j - 1] + 1 
            change = matrix[i - 1][j - 1]

            if str1[i - 1] != str2[j - 1]:
                change += 1

            matrix[i][j] = min(add, delete, change)

            if i > 1 and j > 1 and str1[i - 1] == str2[j - 2] and str1[i - 2] == str2[j - 1]:
                transposition = matrix[i - 2][j - 2] + 1
                matrix[i][j] = min(matrix[i][j], transposition)

    if output:
        print_lev_matrix(matrix, str1, str2)

    return matrix[n][m]
\end{lstlisting}
\end{center}

\clearpage

\textbf{ВЫВОД}

В данном разделе были рассмотрены требования к программному обеспечению, используемые средства реализации, а также приведены листинги кода для вычисления расстояний Левенштейна (на основе рекурсивного и динамического алгоритмов) и Дамерау-Левенштейна (на основе динамического алгоритма).

\clearpage