\begin{center}
    \textbf{ЗАКЛЮЧЕНИЕ}
\end{center}
\addcontentsline{toc}{chapter}{ЗАКЛЮЧЕНИЕ}

Было экспериментально подтверждено различие во временной эффективности рекурсивной и нерекурсивной реализаций выбранных алгоритмов нахождения расстояния между строками при помощи разработаного программного обеспечения на материале замеров процессорного времени выполнения реализаций на различных длинах строк. 

В результате исследований можно сделать вывод о том, что матричная реализация данных алгоритмов, по сравнению с рекурсивной, заметно выигрывает по времени при росте длин строк, но проигрывает по количеству затрачиваемой памяти.

\vspace{5mm}

В ходе выполнения данной лабораторной работы были решены следующие задачи:
\begin{itemize}
	\item реализованы указанные алгоритмы для нахождения расстояния Левенштейна (в рекурсивной и динамической вариации), Дамерау-Левенштейна (в динамической);
	\item проанализированы рекурсивная и динамическая реализации алгоритма Левенштейна, динамические реализации алгоритмов Левенштейна и Дамерау-Левенштейна по затрачиваемым ресурсам (времени и памяти);
	\item описаны и обоснованы полученные результаты в отчете.
\end{itemize}