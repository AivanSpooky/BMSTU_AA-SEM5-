\renewcommand{\bibname}{}
\begin{thebibliography}{6}
\renewcommand{\bibname}{СПИСОК ИСПОЛЬЗОВАННЫХ ИСТОЧНИКОВ}
\begin{center}
    \textbf{\bibname}
\end{center}
\addcontentsline{toc}{chapter}{СПИСОК ИСПОЛЬЗОВАННЫХ ИСТОЧНИКОВ}
    \bibitem{bib0}
    Understanding ‘Winograd Fast Convolution’ [Электронный ресурс]. URL: \url{https://medium.com/@dmangla3/understanding-winograd-fast-convolution-a75458744ff} (дата обращения: 20.09.2024).
	\bibitem{bib1}
	Efficient Winograd Convolution [Электронный ресурс]. URL: \url{https://arxiv.org/pdf/1901.01965} (дата обращения: 20.09.2024).
	\bibitem{bib2}
	C++ Programming Language [Электронный ресурс]. URL: \url{https://devdocs.io/cpp/} (дата обращения: 20.09.2024).
	\bibitem{bib3}
	C++ Date and time utilities [Электронный ресурс]. URL: \url{https://en.cppreference.com/w/cpp/chrono} (дата обращения: 20.09.2024).
    \bibitem{bib4}
	std::clock documentation [Электронный ресурс]. URL: \url{https://en.cppreference.com/w/cpp/chrono/c/clock} (дата обращения: 20.09.2024).
	\bibitem{bib5}
	STM32F303 PDF Documentation [Электронный ресурс]. URL: \url{https://www.st.com/en/microcontrollers-microprocessors/stm32f303/documentation.html} (дата обращения: 20.09.2024).
	\bibitem{bib6}
	Intel® Core™ i5-10300H Processor [Электронный ресурс]. URL: \url{https://ark.intel.com/content/www/us/en/ark/products/201839/intel-core-i5-10300h-processor-8m-cache-up-to-4-50-ghz.html} (дата обращения: 20.09.2024).
\end{thebibliography}