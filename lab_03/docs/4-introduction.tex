\begin{center}
    \textbf{ВВЕДЕНИЕ}
\end{center}
\addcontentsline{toc}{chapter}{ВВЕДЕНИЕ}

Пусть дан словарь, в котором требуется найти элемент. Чтобы успешно его обнаружить или выяснить, что его нет, необходимо применить алгоритм поиска. В данной работе рассматриваются два стандартных подхода для поиска элементов в словаре: полный перебор и бинарный поиск.

\textbf{Цель лабораторной работы} —-- сравнить два алгоритма поиска элемента (полного перебора и бинарного поиска) при работе со словарем. Для достижения поставленной цели необходимо выполнить следующие задачи:

\begin{itemize}
    \item выполнить оценку трудоемкости разрабатываемых алгоритмов;
    \item реализовать алгоритмы нахождения значения в словаре;
    \item выполнить сравнительный анализ сложности двух алгоритмов на основе замеров количества сравнений для каждого элемента массива и данных лучшего и худшего случаев
\end{itemize}