\chapter{Аналитическая часть}

В данном разделе будут рассмотрены алгоритмы нахождения заданного значения в словаре.

\section{Поиск полным перебором}

Для поиска заданного значения, алгоритм начинает перебирать все значения массива с первого до последнего, пока не найдет нужный элемент. Элементы, расположенные в начале массива, будут найдены быстрее тех, что расположены ближе к концу.

\section{Поиск в упорядоченном массиве бинарным поиском}

Для поиска заданного значения, массив должен быть изначально отсортирован. Вводятся левая и правая граница поиска (изначально левая граница --- первый элемент, правая граница --- последний элемент), а также центральный элемент, который сравнивается с искомым значением. Если искомое значение меньше центрального элемента, то правая граница передвигается на место центрального элемента. Если искомое значение больше центрального элемента, то левая граница передвигается на место центрального элемента. Вводится новый центральный элемент для изменившихся границ. Так повторяется, пока искомое значение не будет равно центральному элементу или левая граница станет больше или равна правой.

\vspace{5mm}

\textbf{ВЫВОД}

В данном разделе рассмотрены алгоритмы нахождения заданного значения в массиве.